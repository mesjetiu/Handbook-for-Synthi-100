\selectlanguage{spanish}
\chapter*{Notas sobre la presente edición}

El documento original de este manual pertenece al antiguo Gabinete de Música Electroacústica de Cuenca (España), cuyos fondos están siendo estudiados por el grupo de investigación Fuzzy Gab de la Universidad de Castilla-La Mancha. Dado que el documento está mecanografiado, se ha creado esta edición en \LaTeX{} para mejorar la legibilidad y accesibilidad del contenido, permitiendo además una mejor organización y referencia del mismo.

En esta edición, se han añadido todas las referencias cruzadas necesarias entre figuras, secciones y otros elementos, para facilitar la navegación y comprensión del contenido. Además, se ha añadido un índice de figuras, que no existe en el texto original.

Los textos que aparecen escritos a mano o mecanografiados dentro de las figuras, las cuales han sido escaneadas del documento original, son también buscables dentro del PDF. Esto constituye una característica útil que facilita la navegación y el estudio detallado del documento.

Se han incluido textos [entre corchetes] para identificar aquellos elementos que no están presentes en la versión mecanografiada original, como algunos títulos de ciertas figuras. El índice de contenidos ha sido generado automáticamente en esta versión digital, con lo que puede no coincidir con el índice mecanografiado del documento original, que aquí no se reproduce.

Esperamos que esta edición mejorada sea de gran utilidad para todos los investigadores y entusiastas del Synthi 100.

\vspace{1cm}

\begin{flushright}
	Carlos Arturo Guerra Parra\\
	Investigador en Fuzzy Gab
\end{flushright}

\selectlanguage{english}
\chapter{Functional Principles and Layout, First Switch--On, Brief Checks, Input/Output Connections}

\section{Functional Principles and Layout}
The designer of a synthesizer can go about his work in two quite different ways. He can aim to make the instrument reasonably easy to use by reducing the number of decisions to be made by the user. This means fitting switches or permanent connections between various components in order to reduce the size of the patch (interconnection field with a large number of choices --- like a telephone exchange) which the operator has to control for himself. The trouble is that the fixed decisions reflect what the designer would like, but not necessarily what you would like. Alternatively, he can design a 'no compromise' instrument in which every decision has to be made by the user and a virtually limitless number of configurations are possible. Such machines are not particularly easy to use well, and the larger the machine the more difficult it becomes, but we make no apology for following this second course when we designed the S.100. On the other hand, after first making it difficult, we added a powerful assistant in the shape of the DIGITAL SEQUENCER, which is the most helpful aid to the synthesizer user ever invented. All the same there are 7,200 individual patch points in the S.100 (though we admit that not quite all of them are effectively operational), and thought and experience must be combined if really good work is to result (and the capability is there for such work).

For this reason we strongly recommend any potential user of S.100 to spend a lot of time 'training' on a small synthesizer such as our VCS3, Synthi A or AKS, so that the methods of using matrix board patching, the use of control voltages etc. --- in fact all the basic groundwork of EMS voltage controlled studios, is thoroughly mastered before attempting to use the S.100. In this handbook, therefore, it will be assumed that the reader is already familiar with the operation of the smaller machines.

Because the possibilities are so large, any attempt to provide lists of suitable patches, etc., would be fruitless. We shall explain exactly what each device does, and offer typical examples of its use, but after that it is up to your own musical imagination and your increasing experience with S.100 to build the sounds you think of.

Don't be timid about using the resources of the machine --- or imagine that there is only one way to go about solving a problem. For example, it is easy to settle for only one or two ways of creating a timbre. In fact, even without combining several techniques (which can easily be done) there are at least eight different methods: (1) Addition of shaped waveforms from one source; (2) Additive timbres from a number of oscillators locked by synchronising inputs and tuned to desired partials; (3) Subtractive timbre derivation by manually controlled filters (8--octave filter bank for formants, output filter on each channel); (4) Subtractive timbre derivation by using dynamic voltage controlled filters (of which there are eight); (5) Using the filters themselves as sources, in oscillatory mode, with or without inputs as well; (6) Multiplicative timbre derivation by ring modulation, either with several oscillators (synchronised or not) or one source with frequency doubling and/or halving configurations (S.100 has three ring modulators, which may be cascaded or paralleled in many different ways); (7) Other modulation techniques (a.m. and/or f.m.) by multiple control of filters and oscillators; (8) Digital loops in the sequencer, by which controllable discrete phenomena are speeded up many hundreds of times by increasing clock rate so that they become audio frequency phenomena.


Any of these basic approaches can be further modified by e.g. reverberation (itself voltage controllable) or numerous control influences, some periodic, some asynchronous and some random. Or of course several techniques can be combined to produce timbre changes of great subtlety --- and we haven't even considered any of the ways of incorporating a noise content into the timbre.

This is not an acoustics or musical textbook, so (as with practice on a small synthesizer) we must assume some previous knowledge of the nature of sound and how to manipulate it. Mental analysis of the contents of the mind's ear into likely configurations of actual devices comes with practice. Some will find block diagrams useful before translating these into a practical patch. Others will find it easier to work onto patch dope sheets (you will find a supply of these at Section 8). Others again may incorporate dope sheets into various forms of notation, or work straight on to the machine from an idea.

The above timbre example --- a small fraction of a larger problem --- was intended to encourage rather than daunt, in the sense that there are riches to be found if you take the trouble to find them.

The guiding principles behind the design of the S.100 were (1) to make a synthesizer in which there would be no compromise to versatility; (2) to extend and enhance the control principles already proved highly successful in our small synthesizers --- i.e. matrix patching, bi--polar control voltages, direct coupled circuits which can be used both in sub--audio and audio roles; (3) to include more of everything found in the small synthesizers, but also greatly enhance the specifications of such devices as the Envelope Shaper, include many extra devices such as High Pass--to--Resonating Filters (as well as Low Pass), Envelope Followers, Slew Limiters, Pitch--to--Voltage Converter and Eight Octave Filter Bank; (4) to put the whole of this elaborate network of analogue controls under the command of a digital sequencer, giving an accuracy of performance hitherto undreamed of in analogue synthesizers.

Having decided upon the array of devices we would need, we were faced with nearly 120 inputs and the same number of outputs. Apart from the very high cost of a matrix board this size, it would have been extremely unwieldy, so instead of splitting one board into Signals and Controls (as in VCS3, Synthi A series), we specified two boards of 60 x 60, a Signal Board and a Control Board (left and right respectively). There were some difficult decisions --- borderline cases where a device (e.g. an oscillator) may be a signal or a control. But provision is made for cross--patching, and it is also possible to use jumper leads from patch to patch. The input Amplifiers and Output Channels come up on both boards, and the first four Output Channel inputs appear on the Control Board as well as the Signal Board. Before going any further, and it is not necessary to understand all the devices at this stage, study the boards carefully, and identify all outputs to the patch (horizontal rows) with their appropriate panel controls. Find also the equivalent vertical columns referring to signal inputs (where applicable). The columns on the Control Board do not have any panel controls, because our design policy is to equip all outputs with level controls, and vary input levels from the controlling device. In cases where you wish to feed the same control to different devices at different levels, a selection of different pin resistances can be used (three are supplied as standard, and you can add more yourself).

In Section 2 and 3 you will find all the analogue signal and control devices described, with suggestions for learning their use. Beginning with simple patches, you can build up elaborate configurations in which many parameters are being dynamically controlled.

You may be puzzled to find that three devices --- the Pitch--to--Voltage Converter and the 2 Envelope Followers, have inputs on the Signal Board and Outputs on the Control Board. This is because they are signal--to--control converters, and one of their main functions is in real time transformations of live instrumental inputs. They are described in Section 5.

Functionally, the sequencer is the most elaborate device in S.100, and it is a very powerful control instrument. The programming of analogue control is a matter of delicate adjustment, as you will know from your experience of small synthesizers. On S.100 you may easily have over 100 pins contributing to a patch, and to control all this with accuracy from the two keyboards, the two joysticks, or from the attack--dependent envelope shaper trapezoids or the various free running controls such as the slow oscillators (10, 11, 12) or the random generator, can be a difficult matter when a number of voltages have to change from event to event, or dynamically during them. The sequencer is digital, and operates by storing precise numbers which are converted into equally precise voltages to control to the devices. As you put data into the memory you also hear what you are playing, and on play back you hear it again exactly the same. When adding new voices you can continue to hear any sound previously recorded, and three completely different sets of voltages can be stored, controlling (if you like) quite different types of sound. If you don't want to 'play' your sequence in real time, you don't have to --- you can put in data as slowly and carefully as you like. Mistakes can be singled out and removed or replaced without the least disturbance to any wanted event. Short of full scale computer control (and this is also possible, of course), there is no method of storing voltages which is so accurate and flexible. It makes the S.100, in spite of its huge possibilities of sound production and control, a reasonably docile creature to handle.

The sequencer operating controls are all at the right hand end of the machine --- some on the vertical panel and some below, just above the output channels. The functions of all these controls are described in Section 4, and in Section 6 we describe some more elaborate ways of using the sequencer.

The design of the S.100 does, we believe, allow for any method of work the composer wishes to use. It can be 'played' in real time or programmed slowly and carefully. It can be randomly or accurately controlled (or anything in between). Different types of sound can be set up at leisure one after the other but produced simultaneously. The devices include a very full capability for transforming live instruments, and the ideal studio location is near a performing area so that signals can be processed and sent back to the hall.

\section{First Switch--on and Brief Checks}
The S.100 will have been lined up, and is normally installed, by EMS engineers, and no attempt should be made to alter presets or make other internal adjustments on a new machine without specific instructions from a qualified installer. But the following points may be useful if for any reason you have to install it yourself, or take over a used machine.
\begin{enumerate}
	\item Be quite sure, if the machine has not come direct from us, that the mains supply is correct for the machine. If your supply is 240V, and there is any doubt at all, first obtain a 1KVA 240--110V step--down transformer and run the machine through it. If the machine is already set for 240V the meter and panel lights will be very dim, and the sequencer clock display will not light up. If everything runs normally, you can continue to run the machine through the transformer, but if you prefer to convert it to 240V, all the power supplies must be changed, viz:
	\begin{enumerate}
		\item The power supplies for the S.100 (See technical manual for details)
		\item The mains adjustment for the oscilloscope as fitted.
	\end{enumerate}
\end{enumerate}

\begin{tcolorbox}[colback=yellow!10!white, colframe=red!50!black, title=NOTE]
	THE S.100 IS NOT SUITABLE FOR DC OR BATTERY OPERATION
\end{tcolorbox}

\begin{enumerate}
	\setcounter{enumi}{1}
	\item The stand is normally on castors (though various special stands can be supplied), so the S.100 can be swung out for rear access. Where space considerations make this difficult it may be more convenient to mount it through a partition, giving access from behind, or in some studios it can be free--standing. Ventilation space must be provided above, as the sequencer and the scope both need to dissipate a certain amount of heat. Since one rarely sits at the S.100, some users prefer to raise the whole instrument higher to make it more convenient for standing operation. This also gives clearance to enable the keyboard to be stowed underneath when not in use.
	\item After switching on, make the following checks:
	\begin{enumerate}
		\item Meter dial lamps on (no switch)
		\item Illuminated panel lights (switch near mains switch)
		\item Digital display (Sequencer Event Time). This will come on initially at any number. Pressing CLEAR/RESET or RESET buttons should bring it to zero, where it will stay. It will not run unless patched, so expect no result if you press START.
		\item Envelope Shaper 'ON' lamps. Switch all three shapers to 'Free Run'; DELAY and ON to 0, ATTACK and DELAY about half way. The lamps should flash on and off.
		\item Scope and Frequency Meter. The digital display on the frequency meter should be at zero (except certain types which may be in a time--counting mode). The scope should show two traces (if time base on and tube controls correct).
		\item You can also check all source devices on the scope, but may prefer to do this audibly. A scope check of all oscillators, noise generators and filters (response controls to maximum to put them into oscillation), will, however, familiarise you with patch and panel locations, and scope controls.
	\end{enumerate}
\end{enumerate}

All these simple checks can be done without connecting an output, but temporary listening arrangements are easily made by connecting stereo phones to Pan Outputs 1--4 Left, and 5--8 Left. Put all Pan controls to Left. You can now listen to any output channel, but naturally in order to use the S.100 you must make proper input/output arrangements.

\section{Input/Output Connections}
Look at the back of the lower panel. There is a trough with an access tunnel at each end of the machine, and all leads can be brought in from the rear and concealed neatly in this trough. From left to right you will find:
\begin{itemize}
	\item 4 Pan Outputs
	\item 8 Individual Outputs
	\item 4 External Treatment Sends
	\item 4 External Treatment Returns
	\item 8 Input Amplifiers (sometimes but not always there are ten sockets, Chs. 1 and 2 having extra sockets for MIC. amps).
	\item 2 Option sockets
	\item 2 71--way connectors (one male, one female) to which the whole Control Board is wired. 60 ways on each are used, leaving some spare ways which can be used for other purposes
	\item 1 Keyboard socket
	\item 1 Panel Light switch
	\item 1 Mains Keyswitch (key removable when ON as well as OFF), with main fuse and mains input socket beside it.
\end{itemize}

To take this in order:

	\subsection{Output Channels}
	Figure 1 shows the output arrangements, from which you will see that the Pans are in two separate groups, and can be taken to four outputs if desired. The limitations are that you cannot cross--pan from Outputs 1--4 to Outputs 5--8 or vice versa. The pan outputs are very useful when you have only a limited number of monitor amplifiers and speakers available. If necessary you can hear all 8 channels on two output monitors (same as stereo phones connection mentioned above). With four amplifiers use all four pan outputs. When a full studio mixer is available, complete with its own panning arrangements, it is best to use the individual channel outputs, and this applies also when using multi--track tape recorders unless panning direct from S.100 to a tape recorder. The socket connections are identical on all 12 outputs, and you can change quickly from one type to the other. Figure 2 shows the pin connections for all input and output plugs and sockets, which are balanced lines. Signal levels should be within the range 0--16 dBM. When feeding to an unbalanced source, use earth as pin 1, Signal to pin 3. Pin 2 is left open circuit. (Do not earth). When feeding to the S.100 from unbalanced, separately earthed equipment, connect signal earth to pin 2, signal high to pin 3. (This will avoid hum loops).
	
	\subsection{Treatment Sends and Returns}
	These are general purpose output and input lines which come up on the Signal Board, and are very useful for interfacing with other studio equipment, such as extra filters, reverberation plates, other synthesizers. They can also be used as extra output or input lines. They are individually wired (i.e. there is no special relationship between Send 1 and Return 1), and are fitted with level controls (5K) on the panel. No amplifiers are built in to this facility, but of course an output amplifier can precede a Treatment Send. If extra gain is needed for a Return, one of the Input Amplifiers should be used instead.
	
	\subsection{Input Amplifiers}
	These eight sockets (ten if special microphone amplifier inputs are provided) feed eight identical input amplifiers. They are of line sensitivity (0--16 dBu) or 10VDC Max. for 0.1\% THD) and their direct coupling makes them suitable for signals or controls. Low output microphones need external amplifiers, and in some models Inputs 1 and 2 have special additional microphone sockets with provision for powering these preamplifiers (which are also supplied in these cases).
	
	Inputs can be from a mixer, from other electronic music equipment, from tape recorders, electric instruments, radio, phonograph etc. In some studios it would be most convenient to take all the output, send and return, and input lines to a main studio patch for distribution.
	
	\subsection{Option Sockets}
	These two 8--way sockets are not wired, and are included (like the Option knob at the right of the right hand panel) for the convenience of the user. In any studio there must be room for expansion, and although these sockets may never be used it was felt desirable to include some provision for extra facilities. One obvious use would be to plug in a button box for remote control of tape recorders.
	
	\subsection{Control Board Connectors}
	This plug and socket enable the entire control board to be operated from a remote point. If the S.100 is interfaced with a computer, or its control patch made part of an even larger patch, this facility makes it easy to do. It can, of course, be partly used, in the sense that there might be selected controls you wish to operate remotely (e.g. envelope or sequencer key inputs). There are also a number of unused ways on these plugs, so extra connections can be made.
	
	\subsection{Keyboard Socket}
	To plug in the double 5--octave dynamically proportional keyboard provided.
	
	\subsection{Panel Light and Mains Input}
	The lights under the overhang give working illumination to the lower panels. The mains keyswitch, note, can be removed when the machine is on as well as off. This is valuable when a sequence has been recorded and work interrupted, in case the S.100 is accidentally switched off.
	
	\subsection{Peripherals}
	Apart from the equipment already suggested --- i.e. studio patch, general purpose mixer, tape recorders, monitor amplifiers and speakers --- nothing more is essential to operate the S.100 efficiently. Two excellent measuring devices --- a double beam oscilloscope and a frequency meter --- are incorporated in the machine itself. A digital voltmeter is a useful accessory in certain kinds of work where it is essential to monitor very accurately the voltages being sent to the sequencer or supplied to devices, and is supplied as an in--built facility on certain later models. A reverberation plate is a useful refinement, and this can be connected through the Treatment Sends and Returns. Any studio acquires extra equipment for which there is ample provision in the input/output arrangements of S.100. Many special circumstances can occur, and any competent studio manager will devise suitable methods of control. A single example will show what we mean: The sequencer has a special Key output (Key H --- see Section 6), and it might be a requirement to make this key operate (a) a remote attack in another synthesizer, (b) an electronic stepping device so that successive keys perform different functions, (c) a mechanical relay to e.g. start a tape recorder. All these things can be done --- it is a matter of devising the relatively simple peripheral arrangements you need.
	
	The following two Sections deal respectively with the analogue signal device (sources and treatments) and the analogue control devices. In fact you will have to refer across from one description to another fairly frequently, because it is a practical impossibility to treat each device in isolation.
	

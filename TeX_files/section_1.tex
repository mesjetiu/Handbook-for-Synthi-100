\part{Functional Principles and Layout, First Switch-On, Brief Checks, Input/Output Connections}

\chapter{Functional Principles and Layout}
The designer of a synthesizer can go about his work in two quite different ways. He can aim to make the instrument reasonably easy to use by reducing the number of decisions to be made by the user. This means fitting switches or permanent connections between various components in order to reduce the size of the patch (interconnection field with a large number of choices - like a telephone exchange) which the operator has to control for himself. The trouble is that the fixed decisions reflect what the designer would like, but not necessarily what you would like. Alternatively, he can design a 'no compromise' instrument in which every decision has to be made by the user and a virtually limitless number of configurations are possible. Such machines are not particularly easy to use well, and the larger the machine the more difficult it becomes, but we make no apology for following this second course when we designed the S.100. On the other hand, after first making it difficult, we added a powerful assistant in the shape of the DIGITAL SEQUENCER, which is the most helpful aid to the synthesizer user ever invented. All the same there are 7,200 individual patch points in the S.100 (though we admit that not quite all of them are effectively operational), and thought and experience must be combined if really good work is to result (and the capability is there for such work).

For this reason we strongly recommend any potential user of S.100 to spend a lot of time 'training' on a small synthesizer such as our VCS3, Synthi A or AKS, so that the methods of using matrix board patching, the use of control voltages etc. - in fact all the basic groundwork of EMS voltage controlled studios, 

\newpage

Any of these basic approaches can be further modified by e.g. reverberation (itself voltage controllable) or numerous control influences, some periodic, some asynchronous and some random. Or of course several techniques can be combined to produce timbre changes of great subtlety - and we haven't even considered any of the ways of incorporating a noise content into the timbre.

This is not an acoustics or musical textbook, so (as with practice on a small synthesizer) we must assume some previous knowledge of the nature of sound and how to manipulate it. Mental analysis of the contents of the mind's ear into likely configurations of actual devices comes with practice. Some will find block diagrams useful before translating these into a practical patch. Others will find it easier to work onto patch dope sheets (you will find a supply of these at Section 8). Others again may incorporate dope sheets into various forms of notation, or work straight on to the machine from an idea.

The above timbre example - a small fraction of a larger problem - was intended to encourage rather than daunt, in the sense that there are riches to be found if you take the trouble to find them.

The guiding principles behind the design of the S.100 were (1) to make a synthesizer in which there would be no compromise to versatility; (2) to extend and enhance the control principles already proved highly successful in our small synthesizers - i.e. matrix patching, bi-polar control voltages, direct coupled circuits which can be used both in sub-audio and audio roles; (3) to include more of everything found in the small synthesizers, but also greatly enhance the specifications of such devices as the Envelope Shaper, include many extra devices such as High Pass - to - Resonating Filters (as well as Low Pass), Envelope Followers, Slew Limiters, Pitch - to - Voltage Converter and Eight Octave Filter Bank; (4) to put the whole of this elaborate network of analogue controls under the command of a digital sequencer, giving an accuracy of performance hitherto undreamed of in analogue synthesizers.
